\documentclass[letterpaper,11pt]{article}

\usepackage{fullpage}
\usepackage{amsmath}
\usepackage{amssymb}
\usepackage{amsthm}
\usepackage{graphicx}
\usepackage[font=small]{caption}
\usepackage{subcaption}
\usepackage{hyperref}
\usepackage{listings}
\usepackage{color}
\lstset{ %
language=Matlab,                    % choose the language of the code
basicstyle=\footnotesize\ttfamily,  % the size of the fonts that are used for the code
numbers=none,                       % where to put the line-numbers
numberstyle=\footnotesize,          % the size of the fonts that are used for the line-numbers
stepnumber=1,                       % the step between two line-numbers. If it is 1 each line will be numbered
numbersep=5pt,                      % how far the line-numbers are from the code
backgroundcolor=\color{white},      % choose the background color. You must add \usepackage{color}
showspaces=false,                   % show spaces adding particular underscores
showstringspaces=false,             % underline spaces within strings
showtabs=false,                     % show tabs within strings adding particular underscores
frame=single,                       % adds a frame around the code
tabsize=2,                          % sets default tabsize to 2 spaces
captionpos=t,                       % sets the caption-position to bottom
breaklines=true,                    % sets automatic line breaking
breakatwhitespace=false,            % sets if automatic breaks should only happen at whitespace
escapeinside={\%*}{*)},             % if you want to add a comment within your code
xleftmargin=0.5in,
xrightmargin=0.5in,
columns=fullflexible
}

\newtheorem{lemma}{Lemma}

\DeclareMathOperator{\argmin}{arg\,min}
\newcommand{\Tr}{^\text{T}}
\newcommand{\zeros}{\mathbf{0}}
\newcommand{\ones}{\mathbf{1}}
\newcommand{\reals}{\mathbf{R}}
\newcommand{\complexes}{\mathbf{C}}
\newcommand{\MG}[2]{{\text{#1}(#2)}}
\newcommand{\Un}{\MG{U}{n}}
\newcommand{\On}{\MG{O}{n}}
\newcommand{\SUn}{\MG{SU}{n}}
\newcommand{\SOn}{\MG{SO}{n}}
\newcommand{\qr}{\mathbf{qr}}
\newcommand{\GS}{\mathbf{gs}}

{
\title{Computing the Homology of Classical Groups}
\author{Xavier Falco, Yuze ``Dan'' Huang, Rafael Witten}
\date{\today}
\begin{document}
\maketitle
\thispagestyle{empty}
%\tableofcontents
%\thispagestyle{empty}
\section{Introduction and Explanation of Classical Groups}

In this work we attempt to shed light on the homology of the Classical Groups $\On$, $\SOn$, $\Un$ and $\SUn$. As a reminder, $\On$ is the group of orthogonal matrices viewed as a subgroup of $\reals^{n \times n}$, $\SOn$ is the subgroup of $\On$ with determinant $1$, $\Un$ is the group of unitary matrices viewed as a subgroup of $\complexes^{n\times n}$ and finally $\SUn$ viewed as a subgroup of $\SOn$. In turn, each of these sets inherit the underlying topology of $\reals^{n \times n}$ or $\complexes^{n \times n}$.

\section{Theory}

\subsection{Homology of \On}

$\On$ is intuitively two disjoint copies of $\SOn$. If we define the matrix
\[
  C = \begin{bmatrix}-1 & \zeros\Tr\\ \zeros & I_{n-1} \end{bmatrix}.
\]
and choosing the map $f:\On\to\SOn$ with
\[
  f(X) = \left\{
\begin{array}{ll}
  X & \mbox{if }\det{X} = 1\\
 XC & \mbox{if }\det{X} = -1
\end{array}\right.
\]
we see immediately that $f$ is two to one from $\On$ to $\SOn$. S ince determinant is a continuous function $\On$ is at least two separate components (and we will later show that it is exactly two pieces) and each component is homeomorphic to $\SOn$, so we immediately see that the $k$th Homology Group of $\On$ is the cross-product of two copies of the $k$th Homology Group of $\SOn$.

\subsection{Homology of \SOn}
By counting the number of constraints, we can see immediately that \SOn has dimension $n(n-1)/2$

We prove that $\SOn$ is path-connected. We define $\tt{gs}(X)$ to correspond to the Gram-Schmidt Orthogonalization of an input matrix $X$ and note (the proof is easy) that the Gram-Schmidt Orthogonalization of a non-singular matrix is continuous (although not absolutely-continuous). To define a path from $X \in \SOn$ to $I_n$, we simply choose
\begin{equation}
  p(t) = \GS( (1-t) X + t I_n).
\end{equation}
This is of course a continuous function of $t$ with $p(0) = X$ and $p(1) = I_n$. Critically, $\textbf{det} (p(t))=1$ since the determinant is a continuous function and the determinant of an orthogonal matrix is $\pm 1$ so the determinant can never ``flip".

Trivially \MG{SO}{1} is simply $\{1\}$. Each map in $\MG{SO}{2}$ is defined simply by where it maps $e_1$; where it maps $e_1$ controls where it maps $e_2$. Since $e_1$ can be mapped to any vector of length $1$ we see that $\MG{SO}{2} \simeq S^1$. Finally we make use of the quaternions to capture the group structure of \MG{SO}{3}. If we consider the normalized quaternions (which are vectors in $\reals^4$ with Euclidean norm $1$) we can define a continuous mapping from $S^3$ to $\MG{SO}{3}$ by 
\[
  f(a,b,c,d) = \begin{bmatrix} a^2 + b^2 - c^2 - d^2   & 2bc - 2ad & 2bd + 2ac \\ 2bc+2ad & a^2 - b^2 + c^2- d^2 & 2cd - 2ab \\ 2bd-2ac & 2cd +2ab & a^2-b^2-c^2+d^2 \end{bmatrix}.
\]

This mapping is $2$ to $1$ with $f(x) = f(-x)$. Therefore we consider the projective space $\reals P^3$, which is $S^3$ with $x \sim -x$. It can be shown that the inverse mapping is continuous from $ \MG{SO}{3} \rightarrow \reals P^3$.




\section{Sampling from $\On$, $\SOn$, $\Un$, and $\SUn$}

In this work, will frequently make us of a function for taking the Gram-Schmidt Orthogonalization of a matrix. 
	There is no method in MATLAB that does exactly this, so we wrote a wrapper around $\tt{qr}$ method. 

\begin{lstlisting}[label=gramSchimdt,caption=Computing the Gram-Schmidt Orthogonalization of a input matrix.]
function [X] = gs(M)
[Q,R] = qr(M);
X = Q * diag(sign(diag(R)));
\end{lstlisting}

The method calls $\tt{qr}$ to compute a QR decomposition and then normalizes $Q$ to correspond to the
	 QR decomposition with the diagonal of $R$ real and positive. Since the $\tt{qr}$ decomposition is 
	 unique when applied to non-singular square matrices and $R$ is constrained to have positive 
	 diagonal, the $X$ returned from listing~\ref{gramSchimdt} is the Gram-Schmidt Orthogonalization
	 of the input matrix $M$. 



\subsection{$\On$ and $\SOn$}
The following matlab code samples uniformly from $\On$:

\begin{lstlisting}[label=sample_On,caption=Sampling from $\On$]
function [X] = sample_from_O(n)
[X,R] = gs(randn(n));
\end{lstlisting}

In other words, we take the Gram-Schmidt Orthogonalziation of a matrix whose elements are i.i.d.
  normally distributed.
To sample uniformly from $\SOn$, we sample from $\On$, and then
  ``flip'' the determinant if it is not correct:

\begin{lstlisting}[label=sample_SOn,caption=Sampling from $\SOn$]
function [X] = sample_from_SO(n)
X = sample_from_O(n);
if det(X) < 0
  X(1,:) = -X(1,:);
end
\end{lstlisting}

We will prove that these two algorithms sample uniformly (in the Haar sense)
  from their respective spaces.

\begin{lemma}
Listing~\ref{sample_On} samples uniformly from $\On$, i.e. if $X$
  is the output of listing~\ref{sample_On}, then for any set $S\subset
  \reals^{n\times n}$, $\Pr(X\in S) = \Pr(X\in USV)$ for all $U,V \in \On$.

\begin{proof}
Let $M$ be a random $n\times n$ matrix where the entries $s_{ij}$ are all i.i.d.
  normally distributed.
\begin{align}
\Pr\left(U\GS(M)V\in S\right) &= \Pr\left(\GS(UM)V \in S\right)\\
  &= \Pr\left(\GS(M)V \in S             \right)\\
  &= \Pr\left( (V\Tr\GS(M)\Tr)\Tr \in S \right)\\
  &= \Pr\left( (V\Tr\GS(M))\Tr \in S    \right)\\
  &= \Pr\left((\GS(M))\Tr \in S         \right)\\
  &= \Pr\left(\GS(M) \in S              \right)
\end{align}
\end{proof}
\end{lemma}

\begin{lemma}
Listing~\ref{sample_SOn} induces the Haar measure on $\SOn$.

\begin{proof}
Define the matrix
\[
  C = \begin{bmatrix}-1 & \zeros\Tr\\ \zeros & I_{n-1} \end{bmatrix}.
\]

Define $f:\On\to\SOn$ with
\[
  f(X) = \left\{
\begin{array}{ll}
  X & \mbox{if }\det{X} = 1\\
 XC & \mbox{if }\det{X} = -1
\end{array}\right.
\]
It is clear that listing~\ref{sample_SOn} will sample a matrix $M$ from $\On$,
  and return $f(M)$.

If we let $S \subset \SOn$, then
\begin{align}\label{SOn:uso}
  \Pr\left(f(M) \in S\right)
  &=\Pr\left(M \in S \right) + \Pr\left( MC \in S \right) \\
  &=2\Pr\left(M \in S \right)
\end{align}
because $C$ is orthogonal, and because of the rotational invariance of
  listing~\ref{sample_On}.

Let $\mu_\On$ and $\mu_\SOn$ be the measures induced by
  listings~\ref{sample_On} and \ref{sample_SOn}, respectively.
We showed in (\ref{SOn:uso}) that $\mu_\SOn(S) = 2\mu_\On(S\cap \SOn)$ for
  $S\subset\On$.
Let $S \subset \SOn$, and let $U,V \in \SOn$.
To show that listing~\ref{sample_SOn} induces Haar measure, we note
\begin{align}
  \mu_\SOn(USV) &= 2 \mu_\On \left( USV\cap\SOn \right)\\
  &= 2 \mu_\On \left(U \left(S \cap \SOn \right) V \right)\\
  &= 2 \mu_\On(S \cap \SOn)\\
  &= \mu_\SOn(S).
\end{align}
\end{proof}
\end{lemma}

\subsection{$\Un$ and $\SUn$}

For $\Un$, we have the same idea: generate a random complex matrix, and then
  orthogonalize it. As before, the key is to choose the correct measure to draw
  the initial random complex matrix from.

A proof that the measure induced by listing~\ref{sample_Un} is the Haar
  measure on $\Un$, meaning this sampling method uniformly samples from $\Un$,
  is provided in \cite{Mezzadri2007}.

\begin{lstlisting}[label=sample_Un,caption=Sampling from $\Un$]
function [X] = sample_from_U(n)
X = gs(randn(n) + 1i * randn(n));
end
\end{lstlisting}

To sample uniformly from $\SUn$, we simply force the determinant of the
  output of listing~\ref{sample_Un} to be 1:

\begin{lstlisting}[label=sample_SUn,caption=Sampling from $\SUn$]
function [X] = sample_from_SU(n)
X = sample_from_U(n);
X = X / (det(X)^(1/n));
\end{lstlisting}

\begin{lemma}
Listing~\ref{sample_SUn} induces the Haar measure on $\SUn$.

\begin{proof}
Define $f:\Un\to\SUn$ with $f(X) = (\det X)^{-\frac{1}{n}}X$.
We can also write $f(X) = XC \left((\det X)^{-\frac{1}{n}} \right)$, where $C(x)=xI_n$.
Notice that $C(x)$ commutes with every $n\times n$ matrix for every $x$.
In these terms, listing~\ref{sample_SUn} will sample a matrix $M$ from $\Un$,
  and return $f(M)$.
In other words, if $\mu_\Un$ and $\mu_\SUn$ are the measures induced by
  listings~\ref{sample_Un} and \ref{sample_SUn}, respectively, then
  $\mu_\SUn(S) = \mu_\Un \left(f^{-1}(S) \right)$ for all $S\subset\SUn$.

Let $S\subset \SUn$, and $U,V\in \SUn$.
To show that $\mu_\SUn$ is Haar measure, we see that
\begin{align}
  \mu_\SUn\left(USV\right)  &= \mu_\Un\left(f^{-1}(USV)\right) \\
                            &= \mu_\Un\left(Uf^{-1}(S)V\right) \\
                            &= \mu_\Un\left(f^{-1}(S)\right) \\
                            &= \mu_\SUn\left(S\right).
\end{align}
\end{proof}
\end{lemma}

\section{Experimental Results}

\thispagestyle{empty}
\bibliographystyle{abbrv}
\bibliography{writeup}

\end{document}
}

