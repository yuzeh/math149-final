\documentclass[letterpaper,11pt]{article}

\usepackage{fullpage}
\usepackage{amsmath}
\usepackage{amssymb}
\usepackage{amsthm}
\usepackage{graphicx}
\usepackage[font=small]{caption}
\usepackage{subcaption}
\usepackage{hyperref}

\newtheorem{lemma}{Lemma}

\DeclareMathOperator{\argmin}{arg\,min}
\newcommand{\Tr}{^\text{T}}
\newcommand{\reals}{\mathbf{R}}
\newcommand{\complexes}{\mathbf{C}}
\newcommand{\MG}[2]{\text{#1}(#2)}
\newcommand{\qr}{\mathbf{qr}}

{
\title{Computing the Homology of Matrix groups}
\author{Xavier Falco, Yuze ``Dan'' Huang, Rafael Witten}
\date{\today}
\begin{document}
\begin{titlepage}
\vspace{3in}
\maketitle
\thispagestyle{empty}
\end{titlepage}
\tableofcontents
\thispagestyle{empty}
\section{Sampling from $\MG{O}{n}$, $\MG{SO}{n}$, $\MG{U}{n}$, and $\MG{SU}{n}$}

\subsection{\MG{O}{n} and \MG{SO}{n}}
The following matlab code will sample uniformly from $\MG{O}{n}$:

\begin{verbatim}
function [X] = sample_from_O(n)
[X,R] = qr(randn(n));
\end{verbatim}

In other words, we take the QR decomposition of a matrix whose elements are iid
  normally distributed.
To sample uniformly from $\MG{SO}{n}$, we sample from $\MG{O}{n}$, and then
  ``flip'' the determinant if it is not correct:

\begin{verbatim}
function [X] = sample_from_SO(n)
X = sample_from_O(n);
if det(X) < 0
  X(1,:) = -X(1,:);
end
\end{verbatim}

We will prove that these two algorithms sample uniformly (in the Haar sense)
  from their respective spaces.

\begin{lemma}
The first algorithm [re-word when we have algorithm references] will sample
  uniformly from $\MG{O}{n}$, i.e. if $X$ is the output of the algorithm
  [re-word], then for any set $S\subset \reals^{n\times n}$, $\Pr(X\in S) =
  \Pr(X\in USV)$ for all $U,V \in \MG{O}{n}$.

\begin{proof}
Let $M$ be a random $n\times n$ matrix where the entries $s_{ij}$ are all iid
  normally distributed.
Let $\qr:\reals^{n\times n} \to \MG{O}{n}$ be the ``Q'' in the QR decomposition
  of a matrix.
\begin{align}
\Pr(U\qr(M)V\in S) &= \Pr(\qr(UM)V \in S)\\
  &= \Pr\left(\qr(M)V \in S             \right)\\
  &= \Pr\left( (V\Tr\qr(M)\Tr)\Tr \in S \right)\\
  &= \Pr\left( (V\Tr\qr(M))\Tr \in S    \right)\\
  &= \Pr\left((\qr(M))\Tr \in S         \right)\\
  &= \Pr\left(\qr(M) \in S              \right)\qedhere
\end{align}
\end{proof}
\end{lemma}

\subsection{\MG{U}{n} and \MG{SU}{n}}

\begin{verbatim}
function X = sample_from_U(n)
  [X,R] = qr(randn(n) + 1i * randn(n));
\end{verbatim}
[Reference here]


\thispagestyle{empty}
\bibliographystyle{abbrv}
\bibliography{writeup}

\end{document}
}

