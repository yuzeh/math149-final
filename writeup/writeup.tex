\documentclass[letterpaper,11pt]{article}

\usepackage{fullpage}
\usepackage{amsmath}
\usepackage{amssymb}
\usepackage{amsthm}
\usepackage{graphicx}
\usepackage[font=small]{caption}
\usepackage{subcaption}
\usepackage{hyperref}
\usepackage{listings}
\usepackage{color}
\lstset{ %
language=Matlab,                    % choose the language of the code
basicstyle=\footnotesize\ttfamily,  % the size of the fonts that are used for the code
numbers=none,                       % where to put the line-numbers
numberstyle=\footnotesize,          % the size of the fonts that are used for the line-numbers
stepnumber=1,                       % the step between two line-numbers. If it is 1 each line will be numbered
numbersep=5pt,                      % how far the line-numbers are from the code
backgroundcolor=\color{white},      % choose the background color. You must add \usepackage{color}
showspaces=false,                   % show spaces adding particular underscores
showstringspaces=false,             % underline spaces within strings
showtabs=false,                     % show tabs within strings adding particular underscores
frame=single,                       % adds a frame around the code
tabsize=2,                          % sets default tabsize to 2 spaces
captionpos=t,                       % sets the caption-position to bottom
breaklines=true,                    % sets automatic line breaking
breakatwhitespace=false,            % sets if automatic breaks should only happen at whitespace
escapeinside={\%*}{*)},             % if you want to add a comment within your code
xleftmargin=0.5in,
xrightmargin=0.5in,
columns=fullflexible
}

\newtheorem{lemma}{Lemma}

\DeclareMathOperator{\argmin}{arg\,min}
\newcommand{\Tr}{^\text{T}}
\newcommand{\reals}{\mathbf{R}}
\newcommand{\complexes}{\mathbf{C}}
\newcommand{\MG}[2]{\text{#1}(#2)}
\newcommand{\Un}{\MG{U}{n}}
\newcommand{\On}{\MG{O}{n}}
\newcommand{\SUn}{\MG{SU}{n}}
\newcommand{\SOn}{\MG{SO}{n}}
\newcommand{\qr}{\mathbf{qr}}

{
\title{Computing the Homology of Matrix groups}
\author{Xavier Falco, Yuze ``Dan'' Huang, Rafael Witten}
\date{\today}
\begin{document}
\maketitle
\thispagestyle{empty}
\tableofcontents
\thispagestyle{empty}
\section{Sampling from $\On$, $\SOn$, $\Un$, and $\SUn$}

\subsection{$\On$ and $\SOn$}
The following matlab code will sample uniformly from $\On$:

\begin{lstlisting}[label=sample_On,caption=Sampling from $\On$]
function [X] = sample_from_O(n)
[X,R] = qr(randn(n));
\end{lstlisting}

In other words, we take the QR decomposition of a matrix whose elements are iid
  normally distributed.
To sample uniformly from $\SOn$, we sample from $\On$, and then
  ``flip'' the determinant if it is not correct:

\begin{lstlisting}[label=sample_SOn,caption=Sampling from $\SOn$]
function [X] = sample_from_SO(n)
X = sample_from_O(n);
if det(X) < 0
  X(1,:) = -X(1,:);
end
\end{lstlisting}

We will prove that these two algorithms sample uniformly (in the Haar sense)
  from their respective spaces.

\begin{lemma}
Listing~\ref{sample_On} will sample uniformly from $\On$, i.e. if $X$
  is the output of listing~\ref{sample_On}, then for any set $S\subset
  \reals^{n\times n}$, $\Pr(X\in S) = \Pr(X\in USV)$ for all $U,V \in \On$.

\begin{proof}
Let $M$ be a random $n\times n$ matrix where the entries $s_{ij}$ are all iid
  normally distributed.
Let $\qr:\reals^{n\times n} \to \On$ be the ``Q'' in the QR decomposition
  of a matrix.
\begin{align}
\Pr(U\qr(M)V\in S) &= \Pr(\qr(UM)V \in S)\\
  &= \Pr\left(\qr(M)V \in S             \right)\\
  &= \Pr\left( (V\Tr\qr(M)\Tr)\Tr \in S \right)\\
  &= \Pr\left( (V\Tr\qr(M))\Tr \in S    \right)\\
  &= \Pr\left((\qr(M))\Tr \in S         \right)\\
  &= \Pr\left(\qr(M) \in S              \right)\qedhere
\end{align}
\end{proof}
\end{lemma}

\subsection{$\Un$ and $\SUn$}

For $\Un$, we have the same idea: generate a random complex matrix, and then
  orthogonalize it.
There is one small issue: without forcing constraints on the diagonal of the
  matrix $R$ in the QR decomposition $M=QR$, this decomposition is actually
  {\bf not} unique.
This is important, because some implementations of $\qr$ will not induce
  the Haar measure on $\Un$.
However, if we constrain the diagonal of $R$ to have strictly positive (real)
  entries, then the induced measure will be Haar.

A proof that the measure induced by listing~\ref{sample_Un} is the Haar
  measure on $\Un$, meaning this sampling method uniformly samples from $\Un$,
  is provided in \cite{Mezzadri2007}.

\begin{lstlisting}[label=sample_Un,caption=Sampling from $\Un$]
function [Q] = qr_pos(M)
[Q,R] = qr(M);
Q = Q * diag(sign(diag(R)));
end

function [X] = sample_from_U(n)
X = qr_pos(randn(n) + 1i * randn(n));
end
\end{lstlisting}

To sample uniformly from $\SUn$, we simply force the determinant of the
  output of listing~\ref{sample_Un} to be 1:

\begin{lstlisting}[label=sample_SUn,caption=Sampling from $\SUn$]
function [X] = sample_from_SU(n)
X = sample_from_U(n);
X = X / (det(X)^(1/n));
\end{lstlisting}

\thispagestyle{empty}
\bibliographystyle{abbrv}
\bibliography{writeup}

\end{document}
}

